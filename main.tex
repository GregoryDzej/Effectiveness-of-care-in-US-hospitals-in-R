\documentclass[a4paper]{article}
%\usepackage{simplemargins}

%\usepackage[square]{natbib}
\usepackage{amsmath}
\usepackage{amsfonts}
\usepackage{amssymb}
\usepackage{graphicx}

\begin{document}
\pagenumbering{gobble}

\Large
 \begin{center}
\textbf{Comparative analysis of the 'effectiveness of care' in US hospitals}\\ 

\hspace{10pt}

% Author names and affiliations
\large
Grzegorz Sterkowski$^1$ \\

\hspace{10pt}

\small  
$^1$) Kozminski University\\
32877@kozminski.edu.pl\\

\end{center}

\hspace{10pt}

\normalsize
\textbf{OBJECTIVES:} Describing the effectiveness of the hospital services and optimizing the internal processes is one of the crucial aims of the medical care across the US due to limited resources in health care. The topic we selected for our research is to analyse the level of effectiveness of care provided by different hospitals in comparison to national average level. \textbf{METHODS:} For the analysis, we choose Multinomial Logit model for modelling, given that data set has ordered qualitative data. In total 19 independent variables were available in the data set, out of which only 2 were statistically significant, to explain phenomena about effectiveness of care, given at alpha threshold was equal 0,05. \textbf{RESULTS:} For the hospitals with rating 3(\textit{Ratinglvl1}) odds of hospital effectiveness in comparison to “Same as national average” is 37\%(p = 0.002) lower. Efficient use of medical imaging infrastructure on the level 'Above the national average' will increase odds the hospital effectiveness transit from “Same as national average” higher level by 66\%(p = 0,003). Craig and Uhler’s pseudo R2 statistics of the model is equal to 0,41, which is considerably good fit. \textbf{CONCLUSIONS:} Better quality of rating of the hospital and inclusion of efficient use of medical imaging infrastructure influences effectiveness of the hospital entities in a significant manner. To obtain proper effectiveness, hospital management shall focus on the improvement of their rating, especially those with rating 2 and 3. Furthermore invest more on sufficient imaging infrastructure and proper staff training, accordingly to the budget and efficiency. 
 
\end{document}